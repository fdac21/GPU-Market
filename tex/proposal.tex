
\documentclass[letterpaper, 10 pt, conference]{ieeeconf}  % Comment this line out
                                                          % if you need a4paper
%\documentclass[a4paper, 10pt, conference]{ieeeconf}      % Use this line for a4
                                                          % paper

\IEEEoverridecommandlockouts                              % This command is only
                                                          % needed if you want to
                                                          % use the \thanks command
\overrideIEEEmargins
% See the \addtolength command later in the file to balance the column lengths
% on the last page of the document

\usepackage[utf8]{inputenc}
\usepackage[T1]{fontenc}

% The following packages can be found on http:\\www.ctan.org
%\usepackage{graphics} % for pdf, bitmapped graphics files
%\usepackage{epsfig} % for postscript graphics files
%\usepackage{mathptmx} % assumes new font selection scheme installed
%\usepackage{mathptmx} % assumes new font selection scheme installed
%\usepackage{amsmath} % assumes amsmath package installed
%\usepackage{amssymb}  % assumes amsmath package installed

\title{\LARGE \bf
The GPU Market
}

\author{James hammer% <-this % stops a space
}


\begin{document}



\maketitle
\thispagestyle{empty}
\pagestyle{empty}

\section{Objective}
The objective of this project is to find correlations between historical GPU (Graphical Processing Unit) prices (namely, what percentage they are trading over MSRP) and related factors as discussed below. This will enable more accurate prediction as to when GPU prices can be expected to rise, fall, and stabilize. It may also enable insights to the future of the GPU market, namely, the price of products in the future. A final objective of this project is to rank the related factors in terms of highest to lowest effect on GPU prices, allowing us to empirically attribute the importance of dynamic changes in the market environment.

\section{Motivation}
GPU hardware has never been more sparse, inflated, or sought after than in present times and this trend has no signs of slowing. This project may shed insight on the possible end of the shortages, when to expect future shortages, the best times to buy GPUs, and important signs to be wary of that may adversely effect the GPU market. 

\section{Data}
There will be several sources of data collected which will be compared. These include but are not limited to the following: 

\begin{itemize}
\item historical GPU prices of many GPUs
\item MSRP GPU prices
\item GPU performance metrics
\item availability
\item new hardware releases
\item new software releases
\item general public's interest in GPUs
\item general public's interest in related fields (gaming, modeling rendering, mining)
\item crypto currency trading prices
\item commonality of scalping
\item covid19
\end{itemize}

\section{Responsibilities}
Research - James Hammer

Data collection - James Hammer

Data synthesis - James Hammer

Data comparisons - James Hammer

Data Visualizations - James Hammer

Writing - James Hammer

\section{Milestone Timeline}
Milestone 1 - All relevant data collected

Milestone 2 - All data synthesized to digestible format

Milestone 3 - data comparison workflow created

Milestone 4 - workable results

Milestone 5 - data visualizations created

Milestone 6 - final report, results, and visualizations complete

\section{Challenges}
I expect milestones 1 and 3 to be the most difficult and time consuming. Statistical research will have to be done on the exact nature of the "comparisons" and how to rank their effects on historical GPU trading prices over MSRP. 

Another challenge is managing this project with only 1 team member. 

\section{expected outcome}
I expect that there will be two main forms of data - "inside" factors, or factors determined by the attributes of the GPU itself, and "outside" factors, or factors that are not determined by the attributes of the GPU. I expect that inside factors will not have an effect on the impact on the margin over MSRP that respective GPUs are commonly sold for. I also expect to see that the relative importance (where "importance" is defined as how large an effect each factor has on the GPU price) of each outside factor to rank as follows:

\begin{enumerate}
    \item Covid19
    \item Availability (related to scalping)
    \item All forms of Interests
    \item Crypto trading prices
    \item new releases
\end{enumerate}

These predictions may end up incorrect, but at the very least an expected outcome would be a correct ordering of this list, along with key market features to be aware of when considering investing in a new GPU. Another expected outcome would be a prediction of when GPU prices will return to normal in relation to Covid19.

%\begin{figure}[thpb]
%  \centering
%  %\includegraphics[scale=1.0]{figurefile}
%  \caption{}
%  \label{figurelabel}
%\end{figure}
   

\addtolength{\textheight}{-12cm}   % This command serves to balance the column lengths
                                  % on the last page of the document manually. It shortens
                                  % the textheight of the last page by a suitable amount.
                                  % This command does not take effect until the next page
                                  % so it should come on the page before the last. Make
                                  % sure that you do not shorten the textheight too much.

\end{document}
